\documentclass[11pt]{article}
\usepackage{hyperref}
\setlength{\parskip}{\baselineskip}



\begin{document}

\iffalse
	\begin{titlepage}
    \centering
    \vspace*{30 mm}
    {\huge
    	PREDATOR \& PREY SIMULATION
    	{\scshape multi agent system}
    }
    
    \vspace{30 mm}
    {\Large Subject: Artificial Intelligence Fundamentals}
    
    \vfill
    {\scshape 2014} \\
    {\large Sergiu Terman}
\end{titlepage}

\fi

\section*{Problem Description}

In this problem I am simulating the behavior of multiple \emph{Animals} on a grid. The grid itself represents a forest, a plane with no borders in which a few Animals are placed. This grid has no borders, for instance if I reach position 0 of the x axis -1 is the other side of the grid also known as the maximum x position. The same feature of course also counts for the maximum position plus 1 becoming 0. The simulation happens in turns. In each turn all the Animals will be asked to \emph{make a decision}. Every animal has its own decision making steps, even if the animals in group of predators/preys have a similar behaviour.

There will be \emph{4 types} of Animals. Two predator and two prey type. Two types of species will be able to move (one prey and one predator capable of movement). Each Animal also has a life energy pool, a procreation pool, and an age limit. Animals are able to look to their 8-connected neighboring grid positions for other Animals.

\subsection*{Animal definition}
\begin{enumerate}
\item
Fox (Predator)
	\begin{itemize}
	\item
		Can't move
	\item
		Can eat the prey around him
	\item
		After eating, he remains fed for the next 3 rounds. So he doesn't eat duaring this perioud
	\item
		Can procreate in an empty cell around him, only if the animal is fed
	\end{itemize}
\item
Wolf (Predator)
	\begin{itemize}
	\item
		Attacks the a wolf if there are more than 2 wolves around him
	\item
		The damage done to wolves is in range from $\frac{1}{4}\cdot lifeLimit$ to $\frac{3}{4}\cdot lifeLimit$
	\item
		Can eat the prey around him
	\item
		Can procreate in an empty cell around him, only if the animal is fed
	\item
		If none af actions above were made, wolf moves randomly in a cell around him
	\end{itemize}
\item
Rabbit (Prey)
	\begin{itemize}
	\item
		Can't move
	\item
		If no rabbit around himp he dies of loneliness
	\item
		If there is rabbit around he procreates in an empty cell around him
	\end{itemize}
\item
Deer (Prey)
	\begin{itemize}
	\item
		If there are predators around he escapes by moving in a different cell
	\item
		He escapes only if the empty cell is safe, meaning that no predators around that cell
	\item
		He procreates if a deer is around him
	\item
		If none of the actions above were made then he moves randomly in an empty cell around him
	\end{itemize}
\end{enumerate}

\section*{Implementation}
The implementation of the problem was done in Java programming language. I also have used swing and awt gui libraries for clearer visualization. The source code of the application can be found on this link: \url{https://github.com/xserjjx/Prey-Predator}

Classes used:
\begin{description}
\item
\emph{Animal} is the main abstract class. Here all the shared methods between the animals are implemented. Some of them are:
	\begin{itemize}
	\item
	List<Animal> lookAround(); returns all the animal in the 8 cell neighourhood
	\item
	boolean matingProcess(); does all the mating process and returns true if was done successfully
	\item
	List<Coord> getNeighbourhood(); returns the coordinates of cell around the animal
	\end{itemize}
\item
\emph{Predator} class
\end{description}

\end{document0}


