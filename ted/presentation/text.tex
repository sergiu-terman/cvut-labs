\documentclass{beamer}

\usetheme{Madrid}

\useoutertheme{infolines}
\usecolortheme{beaver}
\setbeamertemplate{itemize items}[default]
\setbeamertemplate{enumerate items}[default]

\newcommand\ListFont{\fontsize{15}{30}\selectfont}

\title[Penetration Testing] % (optional, only for long titles)
{Penetration Testing}
\subtitle{A brief introduction to Penetration Testing}
\author[Sergiu Terman] % (optional, for multiple authors)
{Sergiu Terman}
\institute[FIT CVUT] % (optional)
{CVUT}
\subject{Computer Science}

\begin{document}
\frame{\titlepage}

\section{Introduction}

\begin{frame}
	\frametitle{Things we'll cover today}
	\ListFont
	\begin{itemize}
		\item<2-> What is a pentest
		\item<3-> Terminology
		\item<4->Penetration Test Execution Standards
		\item<5> Tools \& utilities
	\end{itemize}
\end{frame}

\begin{frame}
	\frametitle{Penetration Test (pentest)}
	\begin{overprint}
		\begin{center}
		\includegraphics[height=3cm]{./pentest.png}
		\end{center}
	\end{overprint}

	\fontsize{15}{10}\selectfont
	\begin{alertblock}{}
		\begin{itemize} \itemsep4ex
		\item Attacking a computer system to find it’s vulnerabilities
		\item Many times resumes to gaining access to the system
		\end{itemize}
	\end{alertblock}
\end{frame}

\begin{frame}
	\frametitle{Why need for a pentest?}
	\fontsize{12}{15}\selectfont
	\begin{itemize} \itemsep4ex
	\item<2-> It’s one of the most effective ways to identify weaknesses
	\item<3-> A pentester has to think like a real world (black hat) cracker, so a pentest could reflect the \alert<4->{real life behaviour of an assault}
	\item<5> He has to discover means in which a cracker might compromise the security and deliver damage to the organization
	\end{itemize}
\end{frame}

\begin{frame}
	\frametitle{Types of pentests}
	\begin{itemize} \itemsep8ex
	\item<2-> Overt pentest: (also called white box)
		\begin{itemize}
		\item<3-> The pentester has insider knowledge: the system, it’s infrastructure, etc. (used when time is limited.)
		\end{itemize}

	\item<4-> Covert pentest (also called black box)
		\begin{itemize}
		\item<5-> The pentester has basic or no information whatsoever, except the company name
		\end{itemize}
	\end{itemize}
\end{frame}

\begin{frame}
	\frametitle{Terminology}
	\begin{itemize} \itemsep4ex 
	\item<2-> Exploit 
		\begin{itemize} 
		\item Taking advantage of a flaw within the attacked target. (i.e. SQL injection, configuration errors.)
		\end{itemize}
	\item<3-> Payload
		\begin{itemize} 
		\item Code to be executed on the attacked target. (i.e. and usually a reverse shell or bind shell.)
		\end{itemize}
	\item<4-> Shellcode
		\begin{itemize} 
		\item A piece of code to be run after exploitation, typically written in machine code, usually spawns a shell (hence the name)
		\end{itemize}
	\end{itemize}
\end{frame}

\begin{frame}
	\frametitle{Terminology}
	\begin{itemize}
	\item Vulnerability scanners
		\begin{itemize} \itemsep3ex
			\item<2-> Automated tools to identify known flaws
			\item<3-> First of all - fingerprinting target OS, also it’s services
			\item<4-> Very important in the intelligence gathering step
			\item<5-> Can provide comprehensive vulnerability reports, thus replacing some missing experience
			\item<6> e.g. Retina, Nessus, NeXpose, OpenVAS, Vega, etc
		\end{itemize}
	\end{itemize}
\end{frame}

\begin{frame}
	\frametitle{PTES (Penetration Testing Execution Standard)}
	\begin{itemize} \itemsep3ex
	\item<2-> Pre-engagement interactions. (...and coffee, probably)
	\item<3-> Intelligence gathering. (passive \& active)
	\item<4-> Threat modeling
	\item<5-> Vulnerability analysis
	\item<6-> Exploitation
	\item<7-> Post exploitation
	\item<8> Reporting
	\end{itemize}
\end{frame}

\begin{frame}
	\frametitle{Tools \& utilities}
	\begin{itemize} \itemsep3ex
	\item<2-> Operating systems
		\begin{itemize} \itemsep1ex
		\item<3-> Kali Linux (formerly BackTrack) - based on Debian
		\item<3-> Pentoo - based on Gentoo
		\item<3-> WHAX - based on Slackware
		\end{itemize}
	\item<4-> Frameworks
		\begin{itemize} \itemsep1ex
		\item<5-> Metasploit
		\item<5-> w3af
		\end{itemize}
	\item<6-> Tools
		\begin{itemize} \itemsep1ex
		\item<7-> nmap, netcat, John the Ripper
		\item<7-> tcpdump, Wireshark, upx, etc
		\end{itemize}
	\end{itemize}
\end{frame}

\begin{frame}
	\frametitle{A few words on Metasploit}
	\begin{itemize} \itemsep2ex
	\item<2-> Written entirely in Ruby
	\item<3-> Cross-platform
	\item<4-> As of today, it contains about 1400 different exploits for Windows, Linux, OS X, iOS \& Android, etc
	\item<5-> Uses the modular approach, which makes possible combining different exploits with different payloads
	\item<6-> Highly extensible \& reusable
	\item<7-> Has several useful interfaces (cli, console, armitage)
	\item<8> Free of charge, but commercial versions are also available
	\end{itemize}
\end{frame}

\begin{frame}
	\frametitle{References}
	\begin{itemize} \itemsep3ex
	\item Kennedy D., O’Gorman J., Kearns D., Aharoni M. - Metasploit. The Penetration Tester’s Guide. (2011)
	\item Offensive Security - Metasploit Unleashed \url{http://www.offensive-security.com/metasploit-unleashed}
	\item Penetration Testing Execution Standard \url{http://www.pentest-standard.org/}
	\item Nmap \url{http://nmap.org/}
	\item Metasploit \url{http://www.metasploit.com/}
	\end{itemize}
\end{frame}

\begin{frame}
\begin{center}
	\Huge Thanks for watching
	\begin{exampleblock}{}
		\begin{center}
		{\fontsize{18}{2}\selectfont ``Try Harder''} {\fontsize{11}{2}\selectfont \emph{Offensive Security}}
		\end{center}
	\end{exampleblock}
\end{center}
\end{frame}

\end{document}