\documentclass{beamer}

\usetheme{Madrid}

\useoutertheme{infolines}
\usecolortheme{beaver}
\setbeamertemplate{itemize items}[default]
\setbeamertemplate{enumerate items}[default]

\newcommand\ListFont{\fontsize{15}{30}\selectfont}

\title[Penetration Testing] % (optional, only for long titles)
{Penetration Testing}
\subtitle{A brief introduction to Penetration Testing}
\author[Sergiu Terman] % (optional, for multiple authors)
{Sergiu Terman}
\institute[FIT CVUT] % (optional)
{CVUT}
\subject{Computer Science}

\begin{document}
\frame{\titlepage}

\section{Introduction}

\begin{frame}
	\frametitle{Things we'll cover today}
	\ListFont
	\begin{itemize}
		\item<2-> What is a pentest
		\item<3-> Terminology
		\item<4->Penetration Test Execution Standards
		\item<5> Tools \& utilities
	\end{itemize}
\end{frame}

\begin{frame}
	\frametitle{Penetration Test (pentest)}
	\fontsize{15}{50}\selectfont
	\begin{itemize}
	\item Attacking a computer system to find it’s vulnerabilities
	\item Many times resumes to gaining access to the system
	\end{itemize}
\end{frame}

\begin{frame}
	\frametitle{Why need for a pentest?}
	\fontsize{12}{15}\selectfont
	\begin{itemize} \itemsep4ex
	\item<2-> It’s one of the most effective ways to identify weaknesses
	\item<3-> A pentester has to think like a real world (black hat) cracker, so a pentest could reflect the \alert<4->{real life behaviour of an assault}
	\item<5> He has to discover means in which a cracker might compromise the security and deliver damage to the organization
	\end{itemize}
\end{frame}

\begin{frame}
	\frametitle{Types of pentests}
	\begin{itemize} \itemsep8ex
	\item<2-> Overt pentest: (also called white box)
		\begin{itemize}
		\item<3-> The pentester has insider knowledge: the system, it’s infrastructure, etc. (used when time is limited.)
		\end{itemize}

	\item<4-> Covert pentest (also called black box)
		\begin{itemize}
		\item<5-> The pentester has basic or no information whatsoever, except the company name
		\end{itemize}
	\end{itemize}
\end{frame}

\begin{frame}
	\frametitle{Terminology}
	\begin{itemize}
	\item<2-> Exploit \itemsep4ex
		\begin{itemize} 
		\item Taking advantage of a flaw within the attacked target. (i.e. SQL injection, configuration errors.)
		\end{itemize}
	\item<3-> Payload
		\begin{itemize} 
		\item Code to be executed on the attacked target. (i.e. and usually a reverse shell or bind shell.)
		\end{itemize}
	\item<4-> Shellcode
		\begin{itemize} 
		\item A piece of code to be run after exploitation, typically written in machine code, usually spawns a shell (hence the name)
		\end{itemize}
	\end{itemize}
\end{frame}

\begin{frame}
	\frametitle{Terminology}
	\begin{itemize}
	\item Vulnerability scanners
	\end{itemize}
\end{frame}


\end{document}